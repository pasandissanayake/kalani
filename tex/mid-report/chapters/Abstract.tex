\chapter*{Abstract}
\addcontentsline{toc}{chapter}{Abstract}

\begin{center}
	\vspace{5mm}
	\MakeUppercase{\textbf{Precise Vehicle Localization Using Fusion of Multiple Sensors for Self-Driving}}\\
	\vspace{5mm}
	Group Members: \memberA, \memberB, \memberC, \\ \memberD \\
	\vspace{5mm}
	Supervisor: \supervisorA \\
	\vspace{5mm}
\end{center}

\noindent Keywords: Self-Driving, State Estimation, Localization, Sensor Fusion, Bayesian Filters. \\

This project focuses on creating a mechanism for estimating the state of a self-driving vehicle, including its location, speed and orientation relative to a coordinate frame fixed to earth, using sensors such as \gls{IMU}, \gls{GNSS} receivers, stereo camera pairs and \gls{LiDAR} sensors. The main objective is to deliver a well-documented software stack which includes the state estimator running on \gls{ROS}. The estimator should be capable of providing uninterrupted state estimations with enough accuracy and frequency to facilitate self-driving.

The main drawback observed in current state-of-the-art work is the dependency of the solution on pre-generated highly-detailed maps of different forms, which in-turn reduces the scalability of the solution. This dependency reduces the feasibility of those solutions in the long run due to the fact that it is hard to maintain such highly-detailed maps in midst of constantly and unexpectedly changing environments prevailing in countries such as Sri Lanka. It is the intention of this project to mitigate this dependency through means of improving the state estimation algorithm. We also intend to implement the solution in a modularized architecture to facilitate easy modifications, which in-turn will allow the solution to be used in different applications.

While self-driving is itself a novel concept in Sri Lankan context, this project aims to facilitate the state estimation under constrained resource availability (such as excluding highly-detailed maps, enhanced \gls{GNSS} technologies such as \gls{DGPS} or \gls{RTK} \gls{GPS}, reliable road features such as consistent lane markings and curbs etc.), which is the condition experienced in developing countries like Sri Lanka. 

Other than the self-driving research communities, we expect the outcome of this project will benefit different parties such as robot developers and navigational solution providers, who have similar requirements.
