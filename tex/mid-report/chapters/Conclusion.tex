\chapter{Discussion and Conclusion}
Aim of this project being achieving sub-meter level localization accuracy without depending on pre-built feature maps, we expect to deliver a solution which is a fine-tuned combination of a variety of current state-of-the-art developments. The results have shown that the required accuracy and output rate are feasible goals. 

Summarizing the inferences of the results obtained, we would like to point out the statistical consistency of the system, which is depicted through the consistent error bounds estimated through the sensor fusion process. Furthermore, the noise characteristics of the visual and \gls{LiDAR} odometry algorithms agree the zero-mean white Gaussian assumption, which at the foundation of \gls{ES-EKF}. Improvement in yaw estimation caused by the \gls{ZUPT} measurements is also noteworthy. 

The two main complications at this point of development are, the correlated \gls{GNSS} measurement noise and error covariance estimation of odometry algorithms. Both can be solved in non-optimal ways, which we hesitate to do as it has significant negative impact on the accuracy achievable using a given set of sensors. Apart from the \gls{GNSS} receiver, other sensors happened to have noise characteristics which agree well with the assumptions of the \gls{ES-EKF}. Optimizations required to achieve the expected update rate will be carried out once the filter structure has been finalized. The implementation has been done on \gls{ROS} platform, so that it can be easily integrated with the rest of the modules in a self-driving vehicle.

