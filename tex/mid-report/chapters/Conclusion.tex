\chapter{Discussion and Conclusion}
Aim of this project being achieving sub-meter level localization accuracy without depending on pre-built feature maps, we expect to deliver a solution which is a fine-tuned combination of a variety of current state-of-the-art developments. The results have shown that the required accuracy and output rate is a feasible goal. The two main complications at this point of development are, the correlated \gls{GNSS} measurement noise and error covariance estimation of odometry algorithms. Both can be solved in non-optimal ways, which we hesitate to do as it has significant negative impact on the accuracy achievable using a given set of sensors. Apart from the \gls{GNSS} receiver, other sensors happened to have noise characteristics which agree well with the assumptions of the \gls{ES-EKF}. Optimizations required to achieve the expected update rate will be carried out once the filter structure has been finalized. The implementation has been done on \gls{ROS} platform, so that it can be easily integrated with the rest of the modules in a self-driving vehicle.



